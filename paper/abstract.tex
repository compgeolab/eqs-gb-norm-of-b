Potential field data often require interpolation onto a regular grid at constant height before further analysis. A widely used approach for this is the Equivalent Source Technique, which has been adapted over time to improve the computational efficiency and accuracy of the predictions. However, many of these approaches still face challenges, including border effects in the predictions, reliance on a stabilising parameter and the requirement for regularly gridded data at a constant height. To address these limitations, the Dual-Layer Gradient-Boosted Equivalent Sources (GB EQS) aim to: (1) use the dual-layer approach to improve the accuracy of the predictions and reduce the border effect. (2) Use block-averaging and the GB EQS method to reduce the computational load. (3) Apply Block K-Fold Cross Validation to guide optimal parameter selection for the model. The Dual-Layer GB EQS method is tested on both synthetic datasets and the ICEGRAV dataset to evaluate the methods ability to interpolate and upward continue onto a regular grid at constant height. The Root Mean Square Error is reduced by almost half in comparison to the single-layer approach. Therefore, the Dual-Layer GB EQS does improve the prediction of the total field anomaly and norm of the anomalous magnetic field for large datasets, particularly in the presence of both short- and long-wavelength anomalies.
