Potential field data often require interpolation onto a regular grid at constant height before further analysis. A widely used approach for this is the equivalent sources technique, which has been adapted over time to improve its computational efficiency and accuracy of the predictions. However, many of these approaches still face challenges, including border effects in the predictions or reliance on a stabilising parameter. To address these limitations, we introduce the dual-layer gradient-boosted equivalent sources to: (1) use a dual-layer approach to improve the predictions of both short- and long-wavelength signals, as well as, reduce border effect; (2) use block-averaging and the gradient-boosted equivalent sources method to reduce the computational load; (3) apply block K-Fold cross-validation to guide optimal parameter selection for the model. The proposed method was tested on both synthetic datasets and the ICEGRAV aeromagnetic dataset to evaluate the methods ability to interpolate and upward continue onto a regular grid at constant height, as well as predict the amplitude of the anomalous field from total-field anomaly data. The dual-layer approach proved better compared to the single-layer approach when predicting both short- and long-wavelength signals, particularly in the presence of truncated long-wavelength anomalies. The use of block-averaging and the gradient-boosting method improve the dual-layer approach computationally efficiency, being able to grid over 400,000 data points in under 2 minutes on a moderate workstation computer.