% Template for an EarthArXiv preprint of the manuscript.
% Includes the standard disclaimers and formatting that is required.
%
% This is the document structure. The actual content is written in:
% * abstract.tex: The abstract.
% * abstract-plain.tex: A plain language version of the abstract.
% * content.tex: The actual manuscript text (minus the abstract).
%
%%%%%%%%%%%%%%%%%%%%%%%%%%%%%%%%%%%%%%%%%%%%%%%%%%%%%%%%%%%%%%%%%%%%%%%%%%%%%%%
% Set a class and general configuration
\documentclass[onecolumn,10pt,a4paper]{article}

%%%%%%%%%%%%%%%%%%%%%%%%%%%%%%%%%%%%%%%%%%%%%%%%%%%%%%%%%%%%%%%%%%%%%%%%%%%%%%%
% Set variables with the title, authors, etc.
\newcommand{\Title}{Transforming Total Field Anomaly into Anomalous Magnetic Field: Using Dual-Layer Gradient-Boosted Equivalent Sources}
\newcommand{\TitleShort}{Magnetic Dual-Layer Gradient-Boosted Equivalent Sources}

\newcommand{\Year}{2025}
\newcommand{\SubmittedOn}{2025/04/11}
\newcommand{\PublishedOn}{2024/09/01}

\newcommand{\AuthorShort}{Uppal \textit{et al.}}
\newcommand{\Authors}{%
  India Uppal\textsuperscript{1},
  Leonardo Uieda\textsuperscript{2},
  Vanderlei Coelho Oliveira Jr.\textsuperscript{3},
  Richard Holme\textsuperscript{1}
}
\newcommand{\Email}{I.Uppal@liverpool.ac.uk}
\newcommand{\Corresponding}{%
  Corresponding author: India Uppal <\href{mailto:\Email}{\Email}>
}
\newcommand{\Affiliations}{%
  \textsuperscript{1} University of Liverpool, UK;
  \textsuperscript{2} Universidade de São Paulo, Brazil;
  \textsuperscript{3} Observatório Nacional, Brazil;
}
\newcommand{\AuthorORCIDs}{%
  IU (\href{https://orcid.org/0000-0003-3531-2656}{0000-0003-3531-2656});
  LU (\href{https://orcid.org/0000-0001-6123-9515}{0000-0001-6123-9515});
  VCOJr (\href{https://orcid.org/0000-0002-6338-4086}{0000-0002-6338-4086});
  RH (\href{https://orcid.org/0009-0002-2178-2083}{0009-0002-2178-2083});
}

\newcommand{\Journal}{Geophysical Journal International}
\newcommand{\JournalDOI}{YYYYY/YYYYYYY}
\newcommand{\PreprintDOI}{XXXXX/XXXXXXX}
\newcommand{\ArchiveDOI}{10.5281/zenodo.15120458}
\newcommand{\GitHubRepository}{compgeolab/eqs-gb-norm-of-b}

\newcommand{\Keywords}{%
  Magnetic anomalies: modelling and interpretation; 
  Inverse theory; 
  Antarctica;
}

%%%%%%%%%%%%%%%%%%%%%%%%%%%%%%%%%%%%%%%%%%%%%%%%%%%%%%%%%%%%%%%%%%%%%%%%%%%%%%%
% Import the required packages
\usepackage[utf8]{inputenc}
\usepackage[TU]{fontenc}
\usepackage[english]{babel}
\usepackage{amsmath}
\usepackage{amssymb}
\usepackage{graphicx}
\usepackage{hyperref}
\usepackage{fancyhdr}
\usepackage{orcidlink}
\usepackage{geometry}
\usepackage{booktabs}
\usepackage{microtype}
\usepackage{siunitx}
% To customize the title page
\usepackage{titling}
% For adding multiple authors
\usepackage{authblk}
% improved urls with proper hyphenation
\usepackage{xurl}
% Tweak the look of captions
\usepackage{caption}
% To control the style of section titles
\usepackage{titlesec}
% Import natbib and doi packages
\usepackage[round,authoryear,sort]{natbib}
% show dois as links on references
\usepackage{doi}
% Remove extra space between references
\usepackage{natbibspacing}
% Use a different font
\usepackage[scaled=1.1]{notomath}
% Icons and fonts (requires using xelatex or luatex)
\usepackage{fontawesome5}
\usepackage{academicons}
% Control the font size
\usepackage{anyfontsize}
\usepackage{setspace}
% To get the number of pages in the document
\usepackage{lastpage}
\usepackage{lipsum}
\usepackage{ragged2e}
\usepackage{mdframed}
% To define custom environments
\usepackage{environ}
% To control hyphenation for individual blocks of text
\usepackage{hyphenat}
% To write algorithms as floating objects
\usepackage[longend,linesnumbered,ruled]{algorithm2e}
% To use arrows in math to say "cancel to zero"
\usepackage{cancel}


%%%%%%%%%%%%%%%%%%%%%%%%%%%%%%%%%%%%%%%%%%%%%%%%%%%%%%%%%%%%%%%%%%%%%%%%%%%%%%%
% Configuration of the document

\geometry{%
  left=25mm,
  right=25mm,
  top=18mm,
  bottom=15mm,
  headsep=0mm,
  headheight=0mm,
  footskip=7mm,
  includehead=true,
  includefoot=true
}

% Control line and table row spacing
\onehalfspacing
\renewcommand{\arraystretch}{1.5}

% Set the spacing between bibliography entries (requires natbib)
\setlength{\bibsep}{0pt}

% Custom colors
\definecolor{darkgray}{gray}{0.4}
\definecolor{mediumgray}{gray}{0.5}
\definecolor{lightgray}{gray}{0.9}
\definecolor{mediumblue}{HTML}{2060c2}
\definecolor{lightblue}{HTML}{f7faff}

% Configure captions
\captionsetup{labelfont=bf,font={small,color=darkgray},skip=10pt}

% Make urls use the same font as every other text
\urlstyle{same}

% Configure hyperref and add PDF metadata
\hypersetup{
    colorlinks,
    allcolors=mediumblue,
    pdftitle={\Title},
    pdfauthor={\AuthorShort},
    breaklinks=true,
}

% Configure header and footer
% Inspired by LaPreprint: https://github.com/roaldarbol/LaPreprint
\newcommand{\Separator}{\hspace{3pt}|\hspace{3pt}}
\newcommand{\FooterFont}{\footnotesize\color{mediumgray}}
\pagestyle{fancy}
\fancyhf{}
\lfoot{%
  \FooterFont{}
  \AuthorShort{} (\Year)
  \Separator{}
  \TitleShort
}
\rfoot{%
  \FooterFont{}
  EarthArXiv
  \Separator{}
  \thepage\space of\space \pageref*{LastPage}
}
\renewcommand{\headrulewidth}{0pt}
\renewcommand{\footrulewidth}{1pt}
\preto{\footrule}{\color{lightgray}}
\fancypagestyle{plain}{%
  \fancyhf{}
  \lfoot{%
    \FooterFont{}
    \faCreativeCommons\faCreativeCommonsBy
    \Separator{}
    \textcopyright{} \Year{} The Authors
  }
  \rfoot{%
    \FooterFont{}
    doi:\href{https://doi.org/\PreprintDOI}{\PreprintDOI}
    \Separator{}
    EarthArXiv
    \Separator{}
    \thepage\space of\space \pageref*{LastPage}
  }
}

% Define fancy text boxes
\NewEnviron{summarybox}{%
  \mdfdefinestyle{summarybox_}{%
    leftline=true,
    rightline=false,
    topline=false,
    bottomline=false,
    linewidth=2pt,
    linecolor=mediumblue,
    backgroundcolor=lightblue,
    innertopmargin=12pt,
    innerbottommargin=12pt,
    innerleftmargin=12pt,
    innerrightmargin=12pt,
    skipbelow=5pt,
    skipabove=5pt,
  }
  \newmdenv[style=summarybox_]{summarybox_}
  \begin{summarybox_}
    \footnotesize
    \BODY
  \end{summarybox_}
}

%%%%%%%%%%%%%%%%%%%%%%%%%%%%%%%%%%%%%%%%%%%%%%%%%%%%%%%%%%%%%%%%%%%%%%%%%%%%%%%
\begin{document}

\thispagestyle{plain}
\begin{FlushLeft}
  \begin{spacing}{2}
    {\LARGE\bfseries \Title}
  \end{spacing}
  {\color{lightgray}\hrule height 1.5pt}
  \vspace{0.3cm}
  \Authors
  \\[0.2cm]
  {\footnotesize \Affiliations}
  \newline
  {\footnotesize \Corresponding}
  \\[0.2cm]
  {\footnotesize
    Received in original form on \SubmittedOn.
    %Published in final form on \PublishedOn.
  }
\end{FlushLeft}

\begin{summarybox}
  \noindent
  \textbf{Disclaimer:}
  This is a non-peer reviewed preprint of an article submitted for publication
  in \textit{\Journal{}}. It is available from EarthArXiv at
  \url{https://doi.org/\PreprintDOI}.
  %%%%%%%%%%%%%%%%%%%%%%%%%%%%%%%%%%%%%%%%%%%%%%%%%%%%%%%%%%%%%%%%%%%%%%%%%%%%%
  % Comment the above and uncomment below after publication in a journal
  %%%%%%%%%%%%%%%%%%%%%%%%%%%%%%%%%%%%%%%%%%%%%%%%%%%%%%%%%%%%%%%%%%%%%%%%%%%%%
  % This is a peer-reviewed author-produced postprint of the article
  % ``\AuthorShort{} (\Year). \Title. \textit{\Journal}.
  % doi:\href{https://doi.org/\JournalDOI}{\JournalDOI}''.
  % The postprint is available from EarthArXiv at
  % \url{https://doi.org/\PreprintDOI}.
  %%%%%%%%%%%%%%%%%%%%%%%%%%%%%%%%%%%%%%%%%%%%%%%%%%%%%%%%%%%%%%%%%%%%%%%%%%%%%
  \\[0.25cm]
  \noindent
  \textbf{Open research:}
  The source code used to generate all of the results presented in this
  research can be freely accessed and reused under the terms of an open license.
  You can find it at \url{https://doi.org/\ArchiveDOI} and
  \url{https://github.com/\GitHubRepository}.
  \\[0.2cm]
  \noindent
  \textbf{Keywords:} \Keywords{}
  \\[0.2cm]
  \noindent
  \textbf{ORCID:} \AuthorORCIDs{}
  \\[0.2cm]
  \noindent
  \textbf{\textcopyright{} \Year{} The Authors.}
  Available under the \href{https://creativecommons.org/licenses/by/4.0/}{Creative Commons Attribution 4.0 International License}
  \faCreativeCommons\faCreativeCommonsBy{}.
\end{summarybox}

\section*{\normalsize Abstract}
\begingroup
  \setstretch{1.1} \small Potential field data often require interpolation onto a regular grid at constant height before further analysis. A widely used approach for this is the Equivalent Source Technique, which has been adapted over time to improve the computational efficiency and accuracy of the predictions. However, many of these approaches still face challenges, including border effects in the predictions, reliance on a stabilising parameter and the requirement for regularly gridded data at a constant height. To address these limitations, the Dual-Layer Gradient-Boosted Equivalent Sources (GB EQS) aim to: (1) use the dual-layer approach to improve the accuracy of the predictions and reduce the border effect. (2) Use block-averaging and the GB EQS method to reduce the computational load. (3) Apply Block K-Fold Cross Validation to guide optimal parameter selection for the model. The Dual-Layer GB EQS method is tested on both synthetic datasets and the ICEGRAV dataset to evaluate the methods ability to interpolate and upward continue onto a regular grid at constant height. The Root Mean Square Error is reduced by almost half in comparison to the single-layer approach. Therefore, the Dual-Layer GB EQS does improve the prediction of the total field anomaly and norm of the anomalous magnetic field for large datasets, particularly in the presence of both short- and long-wavelength anomalies.
 \par
\endgroup

%%%%%%%%%%%%%%%%%%%%%%%%%%%%%%%%%%%%%%%%%%%%%%%%%%%%%%%%%%%%%%%%%%%%%%%%%%%%%%%

\section{Introduction}

Here's a text citation of \citet{OliveiraJr2015}
and a another version that is sometimes used \citep{OliveiraJr2015}.

The different gridding methods, such as inverse-distance weights, polynomical fitting, spline functions (Briggs, 1974), often assume potential data are not harmonic functions. However, the total field anomaly is harmonic when the magnitude is much smaller than the magnitude of the geomagnetic field; this is true for most instances in which the total field anomaly is observed. 

Airborne data often have more data points along flight lines and much larger gaps between flight lines. The equivalent source (EQS) technique takes into account the irregularly spaced grids and the variable heights of the observed data. The EQS technique can be used for reduction to the pole, upwards continuation, modelling the lithospheric magnetic field, etc.

The norm of the magnetic field is less dependent on the direction of magnetisation. This is particularly useful for areas with remanent magnetisation, low magnetic latitudes, sources with shallow inclination directions and sources with an unknown magnetised directions (Hidalgo-Gato, et al., 2021; Melo, et al., 2021).

%%%%%%%%%%%%%%%%%%%%%%%%%%%%%%%%%%%%%%%%%%%%%%%%%%%%%%%%%%%%%%%%%%%%%%%%%%%%%%%

\section{Methodology}
The observed total field anomaly, \textcolor{orange}{$\Delta T$}, has N numbrt of data points with positions $(x_i, y_i, z_i)$, where $x_i$ points in the geographic north direction, $y_i$ points in the east direction and $z_i$ points upwards (see orange diamonds in figure XX).


is the difference between the observed magnetic total field intensity vector, $\vec{T}$, and the regional reference field vector, $\vec{F}$:
\begin{equation}
    \textcolor{orange}{\Delta T} = \left\lVert \vec{T} \right\rVert - \left\lVert \vec{F} \right\rVert,
\end{equation}
where the observed $\vec{T}$, is the sum of the anomalous magnetic field vector, $\vec{B}$, and $\vec{F}$:
\begin{equation}
    \textcolor{orange}{\Delta T} = \left\lVert \vec{F} + \vec{B} \right\rVert - \left\lVert \vec{F} \right\rVert.
\end{equation}
Since the $\vec{B}$ is much smaller in comparison to the $\vec{F}$, \deltaT can be approximated as the dot product of the \vec{B} and the regional field direction, \hat{F}:
\begin{equation}
    \textcolor{orange}{\Delta T} \approx  \vec{B} \cdot \hat{F}.
\end{equation}

\subsection{Equivalent Source Technique}
\cite{Cordell1992}'s ‘generalized equivalent sources’ assumes any harmonic function $d_i$ can be approximated by the sum of $m$ discrete point source effects:
\begin{equation}
d_i = \sum_{j=1}^{m} c_j f_{ij}(x,y,z)
\end{equation}

These equations can also be expressed in matrix form as:
\begin{equation}
    \mathbf{d} = \mathbf{Ac}
\end{equation}
where $\mathbf{d}$ is the column vector of $n$ predicted values at the observation points, $\mathbf{c}$ is the column vector of $m$ coefficients (moment magnitude) and $\mathbf{A}$ is the $n \times m$ sensitivity (Jacobian) matrix. By minimising the goal function,
\begin{equation}
    \phi(c) = [\mathbf{d}^0 - \mathbf{Ac}]^T W[\mathbf{d}^0 - \mathbf{Ac}] + \lambda_d \mathbf{c}^T\mathbf{c}
\end{equation}
the values of the source coefficients, $c$, that best fit the observed field values can be obtained using the difference between the observed and the predicted data. If the difference is as close as possible to zero, the observed and predicted data are very similar. Therefore, the smallest,
\begin{equation}
    \phi(c) = min
\end{equation}
gives the best fit. Therefore, the derivetive of $\phi$ is equal to zero.
ADD MATHS!

\subsection{Gradient Boosting}
Estimating the source coefficients that best fit the observed data is computationally demanding. To overcome this problem, \cite{SolerUieda2021} used the gradient boosted method in which the source coefficients are estimated in overlapping windows and carried out iteratively. This gradient boosted method was applied to the shallow layer of equivalent sources.

\subsection{Norm of B Predictions}
The amplitude of the magnetic anomaly $B$ is weakly dependent on the direction of magnetisation compared the total field anomaly $F$ \cite{HidalgoGato2021}. By calculating the norm of the magnetic anomalous field, the data is less dependent on the direction of Earth’s main field and crustal magnetisation.

\subsection{Dual Layer Concept}
Two layers of equivalent sources was used to fit both, the regional magnetic field and the shallow magnetic anomalies. \cite{Li2019} found having an additional deeper layer of equivalent sources reduced the misfit, especially for the long‐wavelength fields. By using two layers, the deeper equivalent source layer can capture the regional, long-wavelength signals, whilst the shallower layer can capture the short-wavelength signals. For this dual layer process, firstly the deeper layer of equivalent sources is calculated to determine the regional field. To determine the depth of this layer, a lower and upper bound of 2.5 and 6 times the distance of the nearest neighbouring data point respectively was used \cite{Dampney1969}. The residual field between the deep equivalent source layer prediction and the data is then used to calculate the shallow equivalent source layer. Adding both the deep and shallow equivalent source layer predictions together creates the final prediction for the data.

\subsection{Block Averaging}
Block averaging the data reduces the computational load and the likelihood of overfitting the data. Due to the nature of aliasing flight line data, block averaging can balance the equivalent sources along and adjacent flight lines to reduce this effect \cite{SolerUieda2021}. Furthermore, by block averaging the data when calculating the deep equivalent sources, only the long-wavelength signals are captured, rather than both short and long wavelength signals. When the block average is too small, more short wavelength signals are captured and not all of the long wavelength signals from the regional field are captured. Too large block averaging, not all the regional signals are captured either. Therefore, different block averaging sizes were calculated to determine the optimum size for capturing only the long wavelength signals. To reduce the computational load when calculating the shallow equivalent sources, block averaging the same size as the grid spacing of the predicted data was applied.

\subsection{K-Fold Cross-Validation}
The model requires manual parameter selection for the damping, depth of equivalent sources and window size for the gradient boosting. In order to select the optimal parameter combination, K-Fold Crosss Validation (K-CV) was used. K-CV is a popular machine learning method often used for model selection. Data is divided into k number of folds, that are as equal as possible in size. One of the k folds is used as testing set and the remaining (k-1) folds are used as training sets. This is repeated iteratively until all k folds have been used as both testing and training sets. The K-CV error estimation is the root mean square error (RMSE) of all the errors from each fold calculation. The parameter combination with the smallest RMSE is selected for the model.

% KCV on slice data, plot pic of training and testing fold 

% https://ieeexplore.ieee.org/abstract/document/5342427
% https://academic.oup.com/biomet/article-abstract/76/3/503/298209?redirectedFrom=fulltext&login=true

\begin{equation}
    \textcolor{orange}{\Delta T} = \left\lVert \vec{T} \right\rVert - \left\lVert \vec{F} \right\rVert
\end{equation}

\begin{equation}
    \textcolor{orange}{\Delta T} = \left\lVert \vec{F} + \vec{B} \right\rVert - \left\lVert \vec{F} \right\rVert \approx  \vec{B} \cdot \hat{F}
\end{equation}

\begin{equation}
\textcolor{orange}{d_i} = \sum_{j=1}^{M}  f_{ij} \textcolor{teal}{c_j}
\end{equation}

\begin{equation}
\textcolor{orange}{\Delta T (x, y, z)} = \sum_{j=1}^{M}  \vec{B_j}(x, y, z) \cdot \hat{F_i} \textcolor{teal}{\left\lVert \vec{m_j} \right\rVert}
\end{equation}

\begin{equation}
\textcolor{orange}{\begin{bmatrix}
    \Delta T_1 \\ \Delta T_2 \\ \vdots \\ \Delta T_N
\end{bmatrix}_{Nx1}} = \begin{bmatrix}
    B_{11} \cdot F_{1} & B_{12} \cdot F_{1} & \hdots & B_{1M} \cdot F_{1} \\
    B_{21} \cdot F_{2} & B_{22} \cdot F_{2} & \hdots & B_{2M} \cdot F_{2} \\
    \vdots & \vdots & \vdots & \vdots \\
    B_{N1} \cdot F_{N} & B_{N2} \cdot F_{N} & \hdots & B_{NM} \cdot F_{N} \\
\end{bmatrix}_{NxM} \textcolor{teal}{\begin{bmatrix}
    \left\lVert \vec{m_1} \right\rVert \\ \left\lVert \vec{m_2} \right\rVert \\ \vdots \\ \left\lVert \vec{m_M} \right\rVert
\end{bmatrix}_{Mx1}}
\end{equation}

\begin{equation}
    \textcolor{orange}{\Delta T (x, y, z)}
\end{equation}

\begin{equation}
  \textcolor{teal}{\left\lVert \vec{m_j} \right\rVert}  
\end{equation}

\begin{equation}
    \vec{B_j}(x, y, z) \cdot \hat{F_i}
\end{equation}

\begin{equation}
    \vec{B} = \frac{C_m}{r^3} [3 ( \hat{m} \cdot \hat{r}) \hat{r} - \hat{m}]
\end{equation}

\begin{equation}
    \textcolor{orange}{\Bar{d}} = \Bar{\Bar{A}} \textcolor{teal}{\Bar{c}}
\end{equation}
\begin{equation}
    \bar{r} = \bar{d^0} - \textcolor{orange}{\bar{d}}
\end{equation}

\begin{equation}
    \phi = \bar{r}^T\bar{r} + \mu \textcolor{teal}{\bar{c}^T\bar{c}}
\end{equation}

\begin{equation}
    \phi (\textcolor{teal}{\bar{c}}) = (\bar{d^0} - \Bar{\Bar{A}} \textcolor{teal}{\Bar{c}})^T (\bar{d^0} - \Bar{\Bar{A}} \textcolor{teal}{\Bar{c}}) + \mu \textcolor{teal}{\bar{c}^T\bar{c}}
\end{equation}

\begin{equation}
    \phi (\textcolor{teal}{\bar{c}}) = \bar{d^0}^T\bar{d^0} - 2 \bar{d^0}^T \Bar{\Bar{A}} \textcolor{teal}{\Bar{c}} + \textcolor{teal}{\Bar{c}}^T \Bar{\Bar{A}}^T \Bar{\Bar{A}} \textcolor{teal}{\Bar{c}} + \mu \textcolor{teal}{\bar{c}^T\bar{c}}
\end{equation}

\begin{equation}
    \nabla_{\textcolor{teal}{\bar{c}}} \phi = 2 \Bar{\Bar{A}}^T \Bar{\Bar{A}} \textcolor{teal}{\Bar{c}} - 2\Bar{\Bar{A}}^T\bar{d^0} + 2\mu \textcolor{teal}{\bar{c}} = \bar{0}
\end{equation}
    
\begin{equation}
    (\Bar{\Bar{A}}^T \Bar{\Bar{A}} + \Bar{\Bar{I}} \mu ) \textcolor{teal}{\bar{c}} = 
    \Bar{\Bar{A}}^T\bar{d^0}
\end{equation}

\begin{equation}
  \textcolor{orange}{\left\lVert \vec{B_i}(x,y,z) \right\rVert} = \sum_{j=1}^{M} \left\lVert \vec{B_j}(x,y,z) \right\rVert \textcolor{teal}{\left\lVert \vec{m_j} \right\rVert}
\end{equation}


%%%%%%%%%%%%%%%%%%%%%%%%%%%%%%%%%%%%%%%%%%%%%%%%%%%%%%%%%%%%%%%%%%%%%%%%%%%%%%%
\section{Synthetic Data Application}

\subsection{With and without dual layer}
* w/o dual layer - doesn't capture all of the regional signal

\subsection{Block average deep sources versus regular grids (for shallow we cite Santi):}
Block average the data to make it smaller and smoother. Show that this works.
Block average source positions instead of regular grid to avoid issues with no-data regions.

\subsection{Single synthetic model}
Use the coordinates from a survey (ICEGRAV) and make a dipole model that is relatively complex but doesn't have to be exactly like the data. Must have regional and shallow sources.


\subsection{Part 1}
Description of the model and data locations.

\subsection{Part 2}
Show the results for our method. Use the block averaging for deep sources + GB for shallow and predict a grid of TFA and |B|. Show maps of the block reduced sources and data, residuals after only deep sources, residuals of the TFA after gradient boosting, grid predictions (deep + shallow) of TFA and |B|.

\subsection{Part 3}
Difference between using the dual-layer or the single shallow layer. Maps showing the differences in TFA and |B|.

\subsection{Part 4}
Block averaging vs regular grid for deep sources. Effects on no-data zones when we use regular grid versus block averaged sources (maps showing the error). Difference in computation time between (1) Block averaged data and sources (2) Blocked averaged sources but original data (3) grid sources and original data (bar plot showing the difference in computation time).

\subsection{Brief discussion (or in a separate section)}

%\begin{figure}[tb]
%\centering
%\includegraphics[width=1\linewidth]{figures/simple-synthetic-data.png}
%\caption{
  %\lipsum[1]
%}
%\label{fig_synthetic_simple_data}
%\end{figure}


%%%%%%%%%%%%%%%%%%%%%%%%%%%%%%%%%%%%%%%%%%%%%%%%%%%%%%%%%%%%%%%%%%%%%%%%%%%%%%%
\section{Real Data Application}

\subsection{Apply the standard method to ICEGRAV}

\subsection{Describe ICEGRAV}

\subsection{Show the original data (TFA points)}

\subsection{Show the location of the deep sources and the block averaged data}

\subsection{Show residuals from deep sources}

\subsection{Show residuals from GB}

\subsection{Grid predictions of TFA and |B|}

\subsection{Brief discussion (or in a separate section)}



%%%%%%%%%%%%%%%%%%%%%%%%%%%%%%%%%%%%%%%%%%%%%%%%%%%%%%%%%%%%%%%%%%%%%%%%%%%%%%%
\section{Conclusion}

\lipsum[1]


%%%%%%%%%%%%%%%%%%%%%%%%%%%%%%%%%%%%%%%%%%%%%%%%%%%%%%%%%%%%%%%%%%%%%%%%%%%%%%%
\section{Open research}

The Python source code used to produce all results and figures presented here
is available at \url{https://github.com/\GitHubRepository} and
\url{https://doi.org/\ArchiveDOI} under the MIT open-source license.

Here we should cite all of the main software used, like Jupyter, numpy, scipy,
matplotlib, Fatiando, etc.

Cite any data sources as well.

%%%%%%%%%%%%%%%%%%%%%%%%%%%%%%%%%%%%%%%%%%%%%%%%%%%%%%%%%%%%%%%%%%%%%%%%%%%%%%%
\section{Acknowledgements}

We are indebted to the developers and maintainers of the open-source software
without which this work would not have been possible.
Acknowledge any non-author contributors to this study.
Statement about funding.

% Thank the editors and reviewers after review.


\bibliographystyle{apalike-doi}
\bibliography{references}

\end{document}
