% Used to set information about the paper that is used in multiple files
%%%%%%%%%%%%%%%%%%%%%%%%%%%%%%%%%%%%%%%%%%%%%%%%%%%%%%%%%%%%%%%%%%%%%%%%%%%%%%%

% Set variables with the title, authors, etc.
\newcommand{\Title}{Transforming Total-Field Anomaly into Anomalous Magnetic Field: Using Dual-Layer Gradient-Boosted Equivalent Sources}
\newcommand{\TitleShort}{Magnetic Dual-Layer Gradient-Boosted Equivalent Sources}

\newcommand{\Year}{2025}
\newcommand{\PreprintOn}{2025/04/25}
\newcommand{\SubmittedOn}{2025/05/16}
\newcommand{\RevisionAOn}{2025/XX/XX}
\newcommand{\PublishedOn}{2023/XX/XX}

\newcommand{\AuthorShort}{Uppal \textit{et al.}}
\newcommand{\Authors}{%
  India Uppal\textsuperscript{1},
  Leonardo Uieda\textsuperscript{2},
  Vanderlei Coelho Oliveira Jr.\textsuperscript{3},
  Richard Holme\textsuperscript{1}
}
\newcommand{\Email}{I.Uppal@liverpool.ac.uk}
\newcommand{\Corresponding}{%
  Corresponding author: India Uppal <\href{mailto:\Email}{\Email}>
}
\newcommand{\Affiliations}{%
  \textsuperscript{1} University of Liverpool, UK;
  \textsuperscript{2} Universidade de São Paulo, Brazil;
  \textsuperscript{3} Observatório Nacional, Brazil;
}
\newcommand{\AuthorORCIDs}{%
  IU (\href{https://orcid.org/0000-0003-3531-2656}{0000-0003-3531-2656});
  LU (\href{https://orcid.org/0000-0001-6123-9515}{0000-0001-6123-9515});
  VCOJr (\href{https://orcid.org/0000-0002-6338-4086}{0000-0002-6338-4086});
  RH (\href{https://orcid.org/0009-0002-2178-2083}{0009-0002-2178-2083});
}

\newcommand{\Journal}{Geophysical Journal International}
\newcommand{\JournalDOI}{YYYYY/YYYYYYY}
\newcommand{\PreprintDOI}{10.31223/X58B1Q}
\newcommand{\ArchiveDOI}{10.5281/zenodo.15120458}
\newcommand{\GitHubRepository}{compgeolab/eqs-gb-norm-of-b}

% From GJI's list of keywords: https://static.primary.prod.gcms.the-infra.com/static/site/gji/document/GJI%20Keywords%202023.pdf?node=ef215d8b2a32e1ee1ffe
\newcommand{\Keywords}{%
  Magnetic anomalies: modelling and interpretation;
  Inverse theory;
  Antarctica;
}
